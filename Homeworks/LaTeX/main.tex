% %%%%%%%%%%%%%%%%%%%%%%%%%%%%%%%%%%%%%%%%%
% % Lachaise Assignment
% % LaTeX Template
% % Version 1.0 (26/6/2018)
% %
% % This template originates from:
% % http://www.LaTeXTemplates.com
% %
% % Authors:
% % Marion Lachaise & François Févotte
% % Vel (vel@LaTeXTemplates.com)
% %
% % License:
% % CC BY-NC-SA 3.0 (http://creativecommons.org/licenses/by-nc-sa/3.0/)
% % 
% %%%%%%%%%%%%%%%%%%%%%%%%%%%%%%%%%%%%%%%%%

% %----------------------------------------------------------------------------------------
% %	PACKAGES AND OTHER DOCUMENT CONFIGURATIONS
% %----------------------------------------------------------------------------------------

\documentclass{article}

%%%%%%%%%%%%%%%%%%%%%%%%%%%%%%%%%%%%%%%%%
% Lachaise Assignment
% Structure Specification File
% Version 1.0 (26/6/2018)
%
% This template originates from:
% http://www.LaTeXTemplates.com
%
% Authors:
% Marion Lachaise & François Févotte
% Vel (vel@LaTeXTemplates.com)
%
% License:
% CC BY-NC-SA 3.0 (http://creativecommons.org/licenses/by-nc-sa/3.0/)
% 
%%%%%%%%%%%%%%%%%%%%%%%%%%%%%%%%%%%%%%%%%

%----------------------------------------------------------------------------------------
%	PACKAGES AND OTHER DOCUMENT CONFIGURATIONS
%----------------------------------------------------------------------------------------

\usepackage{amsmath,amsfonts,stmaryrd,amssymb} % Math packages

\usepackage[ddmmyyyy]{datetime}
\usepackage{enumerate} % Custom item numbers for enumerations

\usepackage[ruled]{algorithm2e} % Algorithms

\usepackage[framemethod=tikz]{mdframed} % Allows defining custom boxed/framed environments

\usepackage{listings} % File listings, with syntax highlighting
\lstset{
  language=C,
  basicstyle=\ttfamily\small\color{black},
  numbers=left,
  numberstyle=\small\color{gray},
  stepnumber=1,
  numbersep=4pt,
  frame=single,
  rulecolor=\color{black},
  framesep=5pt,
  xleftmargin=20pt,
  framexleftmargin=15pt,
  breaklines=true,
  showstringspaces=false,
  keywordstyle=\color{purple}\bfseries,          % parole chiave (if, else, int, return...)
  commentstyle=\color{teal}\itshape,           % commenti
  stringstyle=\color{orange},                   % stringhe
  identifierstyle=\color{black},                % identificatori normali
  emph={size_t, uint32_t, uint64_t, TAS, RTE},            % tipi aggiuntivi
  emphstyle=\color{blue}\bfseries,            % stile tipi aggiuntivi
  morekeywords={uint8_t, uint16_t, uint32_t, uint64_t, int8_t, int16_t, int32_t, int64_t, bool, true, false}, % tipi e parole chiave extra
  morecomment=[l]{//},                           % commenti con //
  morecomment=[s]{/*}{*/},                       % commenti multilinea
}

\lstdefinelanguage{RISCAsm}{
  morekeywords={
    ADDQ, SUM, ANDI, MOVE,MOVEA, MOVEM, DC, DS, sra, 
    lw, sw, li, la, mv, 
    BEQ, BNE, JMP, CMP, jalr, ret, 
    NOP, lui, RTE, RTS, 
    ecall, ebreak, ORG, EQU, JSR, CLR, TAS 
  },
  sensitive=true,
  morecomment=[l]{\*},
  morestring=[b]",
  basicstyle=\ttfamily\small\color{black},
  keywordstyle=\color{purple}\bfseries,
  commentstyle=\color{teal}\itshape,
  stringstyle=\color{orange},
  numbers=left,
  numberstyle=\small\color{gray},
  stepnumber=1,
  numbersep=4pt,
  frame=single,
  rulecolor=\color{black},
  framesep=5pt,
  xleftmargin=20pt,
  framexleftmargin=15pt,
  showstringspaces=false,
  breaklines=true,
}

%----------------------------------------------------------------------------------------
%	DOCUMENT MARGINS
%----------------------------------------------------------------------------------------

\usepackage{geometry} % Required for adjusting page dimensions and margins

\geometry{
	paper=a4paper, % Paper size, change to letterpaper for US letter size
	top=2.5cm, % Top margin
	bottom=3cm, % Bottom margin
	left=2.5cm, % Left margin
	right=2.5cm, % Right margin
	headheight=14pt, % Header height
	footskip=1.5cm, % Space from the bottom margin to the baseline of the footer
	headsep=1.2cm, % Space from the top margin to the baseline of the header
	%showframe, % Uncomment to show how the type block is set on the page
}

%----------------------------------------------------------------------------------------
%	FONTS
%----------------------------------------------------------------------------------------

\usepackage[utf8]{inputenc} % Required for inputting international characters
\usepackage[T1]{fontenc} % Output font encoding for international characters

\usepackage{XCharter} % Use the XCharter fonts
\usepackage{enumitem}
\usepackage{xcolor}
\usepackage{tikz} 
\usetikzlibrary{positioning}

% pseudocode environment %


%----------------------------------------------------------------------------------------
%	COMMAND LINE ENVIRONMENT
%----------------------------------------------------------------------------------------

% Usage:
% \begin{commandline}
%	\begin{verbatim}
%		$ ls
%		
%		Applications	Desktop	...
%	\end{verbatim}
% \end{commandline}

\mdfdefinestyle{commandline}{
	leftmargin=10pt,
	rightmargin=10pt,
	innerleftmargin=15pt,
	middlelinecolor=black!50!white,
	middlelinewidth=2pt,
	frametitlerule=false,
	backgroundcolor=black!5!white,
	frametitle={Command Line},
	frametitlefont={\normalfont\sffamily\color{white}\hspace{-1em}},
	frametitlebackgroundcolor=black!50!white,
	nobreak,
}

% Define a custom environment for command-line snapshots
\newenvironment{commandline}{
	\medskip
	\begin{mdframed}[style=commandline]
}{
	\end{mdframed}
	\medskip
}

%----------------------------------------------------------------------------------------
%	FILE CONTENTS ENVIRONMENT
%----------------------------------------------------------------------------------------

% Usage:
% \begin{file}[optional filename, defaults to "File"]
%	File contents, for example, with a listings environment
% \end{file}

\mdfdefinestyle{file}{
    innertopmargin=1.6\baselineskip,
    innerbottommargin=0.8\baselineskip,
    topline=false, bottomline=false,
    leftline=false, rightline=false,
    leftmargin=2cm,
    rightmargin=2cm,
    singleextra={
        \draw[fill=black!10!white](P)++(0,-1.2em)rectangle(P-|O);
        \node[anchor=north west] at(P-|O){\ttfamily\mdfilename};
        \def\l{3em}
        \draw(O-|P)++(-\l,0)--++(\l,\l)--(P)--(P-|O)--(O)--cycle;
        \draw(O-|P)++(-\l,0)--++(0,\l)--++(\l,0);
    },
}

% Comando per tab a 4 spazi all’interno dell’ambiente file
\newenvironment{file}[1][File]{%
    \medskip
    \newcommand{\mdfilename}{#1}
    % Rende il carattere tab attivo e lo definisce come 4 spazi
    \catcode`\^^I=\active
    \def^^I{\hspace*{4ex}}%
    \begin{mdframed}[style=file]
}{%
    \end{mdframed}
    \medskip
}

%----------------------------------------------------------------------------------------
%	NUMBERED QUESTIONS ENVIRONMENT
%----------------------------------------------------------------------------------------

% Usage:
% \begin{question}[optional title]
%	Question contents
% \end{question}

\mdfdefinestyle{question}{
	innertopmargin=1.2\baselineskip,
	innerbottommargin=0.8\baselineskip,
	roundcorner=5pt,
	nobreak,
	singleextra={%
		\draw(P-|O)node[xshift=1em,anchor=west,fill=white,draw,rounded corners=5pt]{%
		Question \theQuestion\questionTitle};
	},
}

\newcounter{Question} % Stores the current question number that gets iterated with each new question

% Define a custom environment for numbered questions
\newenvironment{question}[1][\unskip]{
	\bigskip
	\stepcounter{Question}
	\newcommand{\questionTitle}{~#1}
	\begin{mdframed}[style=question]
}{
	\end{mdframed}
	\medskip
}

%----------------------------------------------------------------------------------------
%	WARNING TEXT ENVIRONMENT
%----------------------------------------------------------------------------------------

% Usage:
% \begin{warn}[optional title, defaults to "Warning:"]
%	Contents
% \end{warn}

\mdfdefinestyle{warning}{
	topline=false, bottomline=false,
	leftline=false, rightline=false,
	nobreak,
	singleextra={%
		\draw(P-|O)++(-0.5em,0)node(tmp1){};
		\draw(P-|O)++(0.5em,0)node(tmp2){};
		\fill[black,rotate around={45:(P-|O)}](tmp1)rectangle(tmp2);
		\node at(P-|O){\color{white}\scriptsize\bf !};
		\draw[very thick](P-|O)++(0,-1em)--(O);%--(O-|P);
	}
}

% Define a custom environment for warning text
\newenvironment{warn}[1][Warning:]{ % Set the default warning to "Warning:"
	\medskip
	\begin{mdframed}[style=warning]
		\noindent{\textbf{#1}}
}{
	\end{mdframed}
}

%----------------------------------------------------------------------------------------
%	INFORMATION ENVIRONMENT
%----------------------------------------------------------------------------------------

% Usage:
% \begin{info}[optional title, defaults to "Info:"]
% 	contents
% 	\end{info}

\mdfdefinestyle{info}{%
	topline=false, bottomline=false,
	leftline=false, rightline=false,
	nobreak,
	singleextra={%
		\fill[black](P-|O)circle[radius=0.4em];
		\node at(P-|O){\color{white}\scriptsize\bf i};
		\draw[very thick](P-|O)++(0,-0.8em)--(O);%--(O-|P);
	}
}

% Define a custom environment for information
\newenvironment{info}[1][Info:]{ % Set the default title to "Info:"
	\medskip
	\begin{mdframed}[style=info]
		\noindent{\textbf{#1}}
}{
	\end{mdframed}
}
 % Include the file specifying the document structure and custom commands

%----------------------------------------------------------------------------------------
%	ASSIGNMENT INFORMATION
%----------------------------------------------------------------------------------------

\title{Homeworks Algorithms and Data Strucutres} % Title of the assignment

\author{Rocco Lo Russo\\ \texttt{roc.lorusso@studenti.unina.it}} % Author name and email address

\date{Università di Napoli Federico II - DIETI --- \today} % University, school and/or department name(s) and a date

%----------------------------------------------------------------------------------------

\begin{document}

\maketitle % Print the title

%----------------------------------------------------------------------------------------
%	INTRODUCTION
%----------------------------------------------------------------------------------------

\section*{Introduzione} % Unnumbered section
In questo documento verranno sviluppati i due set di Homeworks assegnati per sostenere l'esame.

\section{Homework set 1} \label{sec:homework_1}% Numbered section 
\subsection{Esercizio 1.1} \label{subsec:esercizio1_1}
prova prova
\subsection{Esercizio 1.2} \label{subsec:esercizio1_2}
prova prova
\section{Homework set 2} \label{sec:homework_2}% Numbered section 
\subsection{Esercizio 2.1} \label{subsec:esercizio2_1}
prova prova
\subsection{Esercizio 2.2} \label{subsec:esercizio2_2}
prova prova
\end{document}
% % Math equation/formula
% % \begin{equation}
% % 	I = \int_{a}^{b} f(x) \; \text{d}x.
% % \end{equation}

% % \begin{info} % Information block

% % \end{info}



% %------------------------------------------------


% % Numbered question, with subquestions in an enumerate environment
% % \begin{question}
% % 	Quisque ullamcorper placerat ipsum. Cras nibh. Morbi vel justo vitae lacus tincidunt ultrices. Lorem ipsum dolor sit amet, consectetuer adipiscing elit.

% % 	% Subquestions numbered with letters
% % 	\begin{enumerate}[(a)]
% % 		\item Do this.
% % 		\item Do that.
% % 		\item Do something else.
% % 	\end{enumerate}
% % \end{question}
	
% %------------------------------------------------

% \begin{center}
% 	\begin{minipage}{0.5\linewidth} % Adjust the minipage width to accomodate for the length of algorithm lines
% 		\begin{algorithm}[H]
% 			\KwIn{$(a, b)$, two floating-point numbers}  % Algorithm inputs
% 			\KwResult{$(c, d)$, such that $a+b = c + d$} % Algorithm outputs/results
% 			\medskip
% 			\If{$\vert b\vert > \vert a\vert$}{
% 				exchange $a$ and $b$ \;
% 			}
% 			$c \leftarrow a + b$ \;
% 			$z \leftarrow c - a$ \;
% 			$d \leftarrow b - z$ \;
% 			{\bf return} $(c,d)$ \;
% 			\caption{\texttt{FastTwoSum}} % Algorithm name
% 			\label{alg:fastTwoSum}   % optional label to refer to
% 		\end{algorithm}
% 	\end{minipage}
% \end{center}

% % % Numbered question, with an optional title
% % \begin{question}[\itshape (with optional title)]

% % \end{question}

% mappa della memoria 
% \begin{center}
    
%     \begin{tikzpicture}[scale=0.9]
%         % Rettangolo principale
%         \draw (0,0) rectangle (5,20);
%         \node at (2.4,20.3) {\textbf{Mappa memoria}};
        
%         % vuoto
%         \draw (0,20) rectangle (5,19);
%         \node[right=5pt] at (5,19.5) {\texttt{\$00000000}};
%         \fill[lightgray] (0,20) rectangle (5,19);
%         % INT3
%         \fill[red!20](0,19) rectangle (5,17);
%         \draw (0,19) rectangle (5,18);
%         \node[left=5pt] at (0,18.5) {ISR\_B};
%         \node[right=5pt] at (5,18.5) {\texttt{\$0000006C}};
%         \node at (2.5,18.5) {\$00008700}; 
        
%         % INT4 
%         \draw (0,18) rectangle (5,17);
%         \node[left=5pt] at (0,17.5) {ISR\_C};
%         \node[right=5pt] at (5,17.5) {\texttt{\$00000070}};
%         \node at (2.5,17.5) {\$00008800};

%         % vuoto
%         \draw (0,17) rectangle (5,16);
%         \fill[lightgray] (0,17) rectangle (5,16);
%         %PIA 
%         \fill[yellow!20](0,16) rectangle (5,8);
%         \draw(0,16) rectangle (5,15);
%         \node[right=5pt] at (5,15.5) {\texttt{\$00002004}};
%         \node at (2.5,15.5) {PIABPRA};
%         \draw(0,15) rectangle (5,14);
%         \node[right=5pt] at (5,14.5) {\texttt{\$00002005}};
%         \node at (2.5,14.5) {PIABCRA};
%         \draw(0,14) rectangle (5,13);
%         \node[right=5pt] at (5,13.5) {\texttt{\$00002006}};
%         \node at (2.5,13.5) {PIABPRB};
%         \draw(0,13) rectangle (5,12);
%         \node[right=5pt] at (5,12.5) {\texttt{\$00002007}};
%         \node at (2.5,12.5) {PIABCRB};

%         \draw(0,12) rectangle (5,11);
%         \node[right=5pt] at (5,11.5) {\texttt{\$00002008}};
%         \node at (2.5,11.5) {PIACPRA};
%         \draw(0,11) rectangle (5,10);
%         \node[right=5pt] at (5,10.5) {\texttt{\$00002009}};
%         \node at (2.5,10.5) {PIACCRA};
%         \draw(0,10) rectangle (5,9);
%         \node[right=5pt] at (5,9.5) {\texttt{\$0000200A}};
%         \node at (2.5,9.5) {PIACPRB};
%         \draw(0,9) rectangle (5,8);
%         \node[right=5pt] at (5,8.5) {\texttt{\$0000200B}};
%         \node at (2.5,8.5) {PIACCRB};

%         % vuoto
%         \draw (0,8) rectangle (5,7);
%         \fill[lightgray] (0,8) rectangle (5,7);
        
%         % DATI
%         \fill[green!10] (0,7) rectangle (5,0);
%         \draw(0,7) rectangle (5,6);
%         \node[right=5pt] at (5,6.5) {\texttt{\$00008000}};
%         \node at (2.5,6.5) {AREA DATI};

        
%         % vuoto
%         \draw (0,6) rectangle (5,5);
%         \fill[lightgray] (0,6) rectangle (5,5);
        
%         % CODICE
%         \draw(0,5) rectangle (5,4);
%         \node[right=5pt] at (5,4.5) {\texttt{\$00008200}};
%         \node at (2.5,4.5) {AREA CODICE};

%         % vuoto
%         \draw (0,4) rectangle (5,3);
%         \fill[lightgray] (0,4) rectangle (5,3);

%         % ISRB 
%         \draw(0,3) rectangle (5,2);
%         \node[right=5pt] at (5,2.5) {\texttt{\$00008700}};
%         \node at (2.5,2.5) {ISR\_B};

%         % vuoto
%         \draw (0,2) rectangle (5,1);
%         \fill[lightgray] (0,2) rectangle (5,1);

%         % ISRC
%         \draw(0,1) rectangle (5,0);
%         \node[right=5pt] at (5,0.5) {\texttt{\$00008800}};
%         \node at (2.5,0.5) {ISR\_C};

%         % vuoto 
%         \draw(0,0) rectangle (5,-1);
%         \fill[lightgray](0,0) rectangle (5,-1);
%         % STACK
%         \fill[blue!20](0,-1) rectangle (5,-2);
%         \draw (0,-1) rectangle (5,-2);
%         \node[right=5pt] at (5,-1.5) {\texttt{\$00009000}};
%         \node at (2.5,-1.5) {STACK U-S};

%         \draw(0,-2) rectangle (5,20); % ricalco bordi
%     \end{tikzpicture}
% \end{center}

% % % File contents
% % \begin{file}[hello.py]
% % \begin{lstlisting}[language=Python]
% % #! /usr/bin/python

% % import sys
% % sys.stdout.write("Hello World!\n")
% % \end{lstlisting}
% % \end{file}



% % % Command-line "screenshot"
% % \begin{commandline}
% % 	\begin{verbatim}
% % 		$ chmod +x hello.py
% % 		$ ./hello.py

% % 		Hello World!
% % 	\end{verbatim}
% % \end{commandline}


% % Warning text, with a custom title
% % \begin{warn}[Osservazione:]

% % \end{warn}

% %----------------------------------------------------------------------------------------


% tabella in formato testuale molto cool ben indentata 
% \begin{description}[style=nextline,leftmargin=3.45cm,labelwidth=2.8cm,labelsep=0.6cm,font=\ttfamily\bfseries]
%   \item[fine] Intero che può assumere i valori 0 (il nodo A è in ricezione) o 1 (il nodo A ha terminato la ricezione).

% \end{description}

% \begin{lstlisting}
% c code snippet  
% \end{lstlisting}



% \begin{lstlisting}[language=RISCAsm]
% asm code snippet
% \end{lstlisting}


