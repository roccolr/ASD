% %%%%%%%%%%%%%%%%%%%%%%%%%%%%%%%%%%%%%%%%%
% % Lachaise Assignment
% % LaTeX Template
% % Version 1.0 (26/6/2018)
% %
% % This template originates from:
% % http://www.LaTeXTemplates.com
% %
% % Authors:
% % Marion Lachaise & François Févotte
% % Vel (vel@LaTeXTemplates.com)
% %
% % License:
% % CC BY-NC-SA 3.0 (http://creativecommons.org/licenses/by-nc-sa/3.0/)
% % 
% %%%%%%%%%%%%%%%%%%%%%%%%%%%%%%%%%%%%%%%%%

% %----------------------------------------------------------------------------------------
% %	PACKAGES AND OTHER DOCUMENT CONFIGURATIONS
% %----------------------------------------------------------------------------------------

\documentclass{article}

\input{structure.tex} % Include the file specifying the document structure and custom commands

%----------------------------------------------------------------------------------------
%	ASSIGNMENT INFORMATION
%----------------------------------------------------------------------------------------

\title{Homeworks Algorithms and Data Strucutres} % Title of the assignment

\author{Rocco Lo Russo\\ \texttt{roc.lorusso@studenti.unina.it}} % Author name and email address

\date{Università di Napoli Federico II - DIETI --- \today} % University, school and/or department name(s) and a date

%----------------------------------------------------------------------------------------

\begin{document}

\maketitle % Print the title

%----------------------------------------------------------------------------------------
%	INTRODUCTION
%----------------------------------------------------------------------------------------

\section*{Introduzione} % Unnumbered section
In questo documento verranno sviluppati i due set di Homeworks assegnati per sostenere l'esame.

\section{Homework set 1} \label{sec:homework_1}% Numbered section 
\subsection{Esercizio 1.1} \label{subsec:esercizio1_1}
\textbf{Traccia}:

\noindent
\textit{Per ognuna delle seguenti affermazioni, si dica se essa è sempre vera, mai vera, o a volte 
vera, per funzioni asintoticamente non-negative. Se la si considera sempre vera o mai vera, 
si spieghi il perché. Se è a volte vera, si dia un esempio per cui è vera e uno per cui è falsa.}
\begin{itemize}
    \item $f(n) = O(f(n)^2)$;
    \item $f(n) + O(f(n)) =  \Theta(f(n))$;
    \item $ f(n) = \Omega(g(n))$ e $f(n) = o(g(n))$
\end{itemize}
\vspace{\baselineskip}

\noindent
\textbf{Soluzione}: 

\noindent
$f(n) = O(f(n)^2) $ è un'affermazione vera a volte: infatti $f(n) = O(f(n)^2) \iff f(n) \le cf(n)^2 $ per qualche costante $ c>0 $ e per qualche $n > n_0$. Poichè la funzione f(n) è definita asintoticamente \textit{non negativa}, possiamo assumere che per n sufficientemente grande la funzione assumerà soltanto valori o positivi o nulli. 

\noindent
Nel caso in cui $f(n) = 0$, la disuguaglianza è verificata perchè $ 0 \le c \cdot 0^2$ è vero sempre. 

\noindent
Nel caso in cui $f(n) > 0$:

\begin{equation} \label{eq:proof_1_1_1}
\begin{aligned}
f(n) &\le cf(n)^2 \implies f(n) \ge \frac{1}{c} \text{,  } \forall n \ge n_0
\end{aligned}
\end{equation}

\noindent
La disuguaglianza nel secondo caso è verificata solo se la funzione $f(n)$ risulta, per n sufficientemente grande, maggiore di $\frac{1}{c}$, ovvero se risulta limitata inferiormente da una costante strettamente maggiore di 0, come riportato nel passaggio (\ref{eq:proof_1_1_1}).
Un esempio di funzione asintoticamente non negativa che soddisfa la definizione è $f(n) = n$, mentre un esempio di funzione asintoticamente non negativa che non soddisfa la definizione è $f(n) = \frac{1}{n}$.
\vspace{\baselineskip}

\noindent
Per quanto riguarda la seconda affermazione, $f(n) + O(f(n)) =  \Theta(f(n))$, dimostriamo che è sempre vera: il generico elemento $h(n) \in \{f(n) + O(n)\}$ possiamo scriverlo come $h(n) = f(n) + g(n)$, con $g(n) \in O(n)$; I passaggi illustrati in  (\ref{eq:proof_1_1_2_1}) dimostrano che $f(n) + O(f(n)) = \Theta(f(n))$ $\forall n \ge n_0$, mentre i passaggi illustrati in (\ref{eq:proof_1_1_2_2}) dimostrano che $f(n) + O(f(n)) = \Omega(f(n))$ $\forall n \ge n_0$. 

\begin{equation} \label{eq:proof_1_1_2_1}
\begin{aligned}
g(n) \le cf(n) \implies h(n) &= f(n) + g(n) \\ &\le f(n) + cf(n) \\ 
&\le (1+c)f(n) 
\end{aligned}
\end{equation}

\begin{equation} \label{eq:proof_1_1_2_2}
\begin{aligned}
g(n) \ge 0 \implies f(n) + g(n) \ge f(n)
\end{aligned}
\end{equation}

\noindent
Possiamo concludere che il generico elemento $h(n) \in \{f(n)+O(f(n))\}$ è sia un elemento di $O(f(n))$ che un elemento di $\Omega(f(n))$, ragione per cui $\{f(n)+O(f(n))\} \subseteq {\Theta(f(n))} \implies f(n) + O(f(n)) = \Theta(f(n))$.


\subsection{Esercizio 1.2} \label{subsec:esercizio1_2}
prova prova
\section{Homework set 2} \label{sec:homework_2}% Numbered section 
\subsection{Esercizio 2.1} \label{subsec:esercizio2_1}
prova prova
\subsection{Esercizio 2.2} \label{subsec:esercizio2_2}
prova prova
\end{document}
% % Math equation/formula
% % \begin{equation}
% % 	I = \int_{a}^{b} f(x) \; \text{d}x.
% % \end{equation}

% % \begin{info} % Information block

% % \end{info}



% %------------------------------------------------


% % Numbered question, with subquestions in an enumerate environment
% % \begin{question}
% % 	Quisque ullamcorper placerat ipsum. Cras nibh. Morbi vel justo vitae lacus tincidunt ultrices. Lorem ipsum dolor sit amet, consectetuer adipiscing elit.

% % 	% Subquestions numbered with letters
% % 	\begin{enumerate}[(a)]
% % 		\item Do this.
% % 		\item Do that.
% % 		\item Do something else.
% % 	\end{enumerate}
% % \end{question}
	
% %------------------------------------------------

% \begin{center}
% 	\begin{minipage}{0.5\linewidth} % Adjust the minipage width to accomodate for the length of algorithm lines
% 		\begin{algorithm}[H]
% 			\KwIn{$(a, b)$, two floating-point numbers}  % Algorithm inputs
% 			\KwResult{$(c, d)$, such that $a+b = c + d$} % Algorithm outputs/results
% 			\medskip
% 			\If{$\vert b\vert > \vert a\vert$}{
% 				exchange $a$ and $b$ \;
% 			}
% 			$c \leftarrow a + b$ \;
% 			$z \leftarrow c - a$ \;
% 			$d \leftarrow b - z$ \;
% 			{\bf return} $(c,d)$ \;
% 			\caption{\texttt{FastTwoSum}} % Algorithm name
% 			\label{alg:fastTwoSum}   % optional label to refer to
% 		\end{algorithm}
% 	\end{minipage}
% \end{center}

% % % Numbered question, with an optional title
% % \begin{question}[\itshape (with optional title)]

% % \end{question}

% mappa della memoria 
% \begin{center}
    
%     \begin{tikzpicture}[scale=0.9]
%         % Rettangolo principale
%         \draw (0,0) rectangle (5,20);
%         \node at (2.4,20.3) {\textbf{Mappa memoria}};
        
%         % vuoto
%         \draw (0,20) rectangle (5,19);
%         \node[right=5pt] at (5,19.5) {\texttt{\$00000000}};
%         \fill[lightgray] (0,20) rectangle (5,19);
%         % INT3
%         \fill[red!20](0,19) rectangle (5,17);
%         \draw (0,19) rectangle (5,18);
%         \node[left=5pt] at (0,18.5) {ISR\_B};
%         \node[right=5pt] at (5,18.5) {\texttt{\$0000006C}};
%         \node at (2.5,18.5) {\$00008700}; 
        
%         % INT4 
%         \draw (0,18) rectangle (5,17);
%         \node[left=5pt] at (0,17.5) {ISR\_C};
%         \node[right=5pt] at (5,17.5) {\texttt{\$00000070}};
%         \node at (2.5,17.5) {\$00008800};

%         % vuoto
%         \draw (0,17) rectangle (5,16);
%         \fill[lightgray] (0,17) rectangle (5,16);
%         %PIA 
%         \fill[yellow!20](0,16) rectangle (5,8);
%         \draw(0,16) rectangle (5,15);
%         \node[right=5pt] at (5,15.5) {\texttt{\$00002004}};
%         \node at (2.5,15.5) {PIABPRA};
%         \draw(0,15) rectangle (5,14);
%         \node[right=5pt] at (5,14.5) {\texttt{\$00002005}};
%         \node at (2.5,14.5) {PIABCRA};
%         \draw(0,14) rectangle (5,13);
%         \node[right=5pt] at (5,13.5) {\texttt{\$00002006}};
%         \node at (2.5,13.5) {PIABPRB};
%         \draw(0,13) rectangle (5,12);
%         \node[right=5pt] at (5,12.5) {\texttt{\$00002007}};
%         \node at (2.5,12.5) {PIABCRB};

%         \draw(0,12) rectangle (5,11);
%         \node[right=5pt] at (5,11.5) {\texttt{\$00002008}};
%         \node at (2.5,11.5) {PIACPRA};
%         \draw(0,11) rectangle (5,10);
%         \node[right=5pt] at (5,10.5) {\texttt{\$00002009}};
%         \node at (2.5,10.5) {PIACCRA};
%         \draw(0,10) rectangle (5,9);
%         \node[right=5pt] at (5,9.5) {\texttt{\$0000200A}};
%         \node at (2.5,9.5) {PIACPRB};
%         \draw(0,9) rectangle (5,8);
%         \node[right=5pt] at (5,8.5) {\texttt{\$0000200B}};
%         \node at (2.5,8.5) {PIACCRB};

%         % vuoto
%         \draw (0,8) rectangle (5,7);
%         \fill[lightgray] (0,8) rectangle (5,7);
        
%         % DATI
%         \fill[green!10] (0,7) rectangle (5,0);
%         \draw(0,7) rectangle (5,6);
%         \node[right=5pt] at (5,6.5) {\texttt{\$00008000}};
%         \node at (2.5,6.5) {AREA DATI};

        
%         % vuoto
%         \draw (0,6) rectangle (5,5);
%         \fill[lightgray] (0,6) rectangle (5,5);
        
%         % CODICE
%         \draw(0,5) rectangle (5,4);
%         \node[right=5pt] at (5,4.5) {\texttt{\$00008200}};
%         \node at (2.5,4.5) {AREA CODICE};

%         % vuoto
%         \draw (0,4) rectangle (5,3);
%         \fill[lightgray] (0,4) rectangle (5,3);

%         % ISRB 
%         \draw(0,3) rectangle (5,2);
%         \node[right=5pt] at (5,2.5) {\texttt{\$00008700}};
%         \node at (2.5,2.5) {ISR\_B};

%         % vuoto
%         \draw (0,2) rectangle (5,1);
%         \fill[lightgray] (0,2) rectangle (5,1);

%         % ISRC
%         \draw(0,1) rectangle (5,0);
%         \node[right=5pt] at (5,0.5) {\texttt{\$00008800}};
%         \node at (2.5,0.5) {ISR\_C};

%         % vuoto 
%         \draw(0,0) rectangle (5,-1);
%         \fill[lightgray](0,0) rectangle (5,-1);
%         % STACK
%         \fill[blue!20](0,-1) rectangle (5,-2);
%         \draw (0,-1) rectangle (5,-2);
%         \node[right=5pt] at (5,-1.5) {\texttt{\$00009000}};
%         \node at (2.5,-1.5) {STACK U-S};

%         \draw(0,-2) rectangle (5,20); % ricalco bordi
%     \end{tikzpicture}
% \end{center}

% % % File contents
% % \begin{file}[hello.py]
% % \begin{lstlisting}[language=Python]
% % #! /usr/bin/python

% % import sys
% % sys.stdout.write("Hello World!\n")
% % \end{lstlisting}
% % \end{file}



% % % Command-line "screenshot"
% % \begin{commandline}
% % 	\begin{verbatim}
% % 		$ chmod +x hello.py
% % 		$ ./hello.py

% % 		Hello World!
% % 	\end{verbatim}
% % \end{commandline}


% % Warning text, with a custom title
% % \begin{warn}[Osservazione:]

% % \end{warn}

% %----------------------------------------------------------------------------------------


% tabella in formato testuale molto cool ben indentata 
% \begin{description}[style=nextline,leftmargin=3.45cm,labelwidth=2.8cm,labelsep=0.6cm,font=\ttfamily\bfseries]
%   \item[fine] Intero che può assumere i valori 0 (il nodo A è in ricezione) o 1 (il nodo A ha terminato la ricezione).

% \end{description}

% \begin{lstlisting}
% c code snippet  
% \end{lstlisting}



% \begin{lstlisting}[language=RISCAsm]
% asm code snippet
% \end{lstlisting}


